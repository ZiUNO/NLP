\documentclass[UTF8]{ctexart}
\title{A Neural Probabilistic Language Model}
\author{ZiUNO}
\date{}
\usepackage{amsmath}
\usepackage{graphicx}
\usepackage{fancyhdr}
\pagestyle{fancy}
\lhead{\author}
\chead{\date}
\rhead{ZiUNO}
\lfoot{}
\cfoot{\thepage}
\rfoot{}
\renewcommand{\headrulewidth}{0.4pt}
\renewcommand{\headwidth}{\textwidth}
\renewcommand{\footrulewidth}{0pt}
\usepackage{setspace}
\onehalfspacing
\begin{document}
\maketitle
\begin{abstract}
基于统计学的语言模型的目标是获得词序列的联合概率分布。然而,由于维度诅咒的存在,这在本质上十分困难,因此本文尝试从问题自身寻找解决方法。在本文内容中,同时解决了(1)词分布表示和(2)词序列概率分布问题。本文在两个文本数据集上进行实验验证,相较于最先进的三元语言模型,本文中使用神经网络的概率分布,效果取得显著提升。
\end{abstract}
\section{引言}
\par{在词序列中,给定前面的词的情况下,下一个词的条件概率可用于表示基于统计学的语言模型,即$P(w_{1}^T)=\prod_{t=1}^TP(w_{t}|w_{1}^{t-1})$。其中,$w_{t}$表示第$t$个词,子序列$w_{i}^j=(w_{i},w_{i+1},\cdots,w_{j-1},w_{j})$。}
\par{事实上,在词序列中,顺序近的词在统计学上更具有依赖性。因此,对于每个上下文,对下一个词的概率分布采用$n-gram$模型,例如,结合前$n-1$个词,则$P(w_{t}|w_{1}^{t-1})\approx P(w_{t}|w_{t-n+1}^{t-1})$。然而,这只适用于在训练集中出现或词频高的词组合,对于在训练集中未出现的$n$个词的新组合,可以简单地通过使用更小的上下文来解决,如退化三元模型或者插值三元模型。针对从训练集中出现的词序列推广到新的词序列问题,可以在足够短的上下文中,例如,将训练集中词频高的长度为1、2或3的词片段进行“粘合”,并获取该长序列的概率。然而,这种情况下存在以下缺陷:}
\begin{enumerate}
  \item 未考虑长度远远超过1或2个词的上下文;
  \item 未考虑词语之间的相似性。
\end{enumerate}
\subsection{解决维度诅咒}
本文方法思想总结如下:
\begin{enumerate}
  \item 将单词表中每个词映射到一个分布“特征向量”上,因此可生成词间相似度;
  \item 使用词的特征向量表示词序列的联合概率分布;
  \item 同时获得词特征向量和分布中的参数。
\end{enumerate}
\section{两种架构}
\par{训练集为词序列$w_{1}\cdots w_{T}$($w_{t}\in V$),其中,$V$表示有限词汇集。本文目标为获取模型满足$f(w_{t},\cdots,w_{t-n})=\widehat{P}(w_{t}|w_{1}^{t-1})$,并使得该模型对样本外数据可以得到高似然率。实验中使用$1/\widehat{P}(w_t|w_{1}^{t-1})$的几何平均表示困惑度(perplexity),即平均负对数似然率的指数。模型的唯一限制条件为:$\forall w_{1}^{t-1}, \sum _{i=1}^{|V|}f(i,w_{t-1},w_{t-n})=1$。}
\par{模型基础构成如上所述。为了该模型加速和推广模型将进行如下改良,接下来将$f(w_{t},\cdots,w_{t-n})=\widehat{P}(w_{t}|w_{1}^{t-1})$分为两部分:}
\begin{enumerate}
  \item 映射$C:V\rightarrow C(i)$,其中,$C(i)\in R^{m}$,$C$的大小为$|V| \times m$。
  \item 使用$C$表示的词概率分布。以下为两种构想:
  \begin{enumerate}
    \item 直连架构:根据以前的词得到下一个词的概率分布。
    \begin{enumerate}
      \item $input$:$(w_{t-n},\cdots,w_{t-1})$
      \item $C(i)$:$(C(w_{t-n}),\cdots,C(w_{t-1}))$
      \item $h_{i}$:$h(i,C(w_{t-1}),\cdots,C(w_{t-n}))$
      \item $output$:$softmax(h)_{i}$
    \end{enumerate}
    \item 循环架构:根据当前所有词(下一个词为所有可能词)获得概率分布。
    \begin{enumerate}
      \item $input$:$(w_{t-n},\cdots,w_{t-1},i)$
      \item $C(i)$:$(C(w_{t-n}),\cdots,C(w_{t-1}),C(i))$
      \item $h_{i}$:$h(C(i),C(w_{t-1}),\cdots,C(w_{t-n}))$
      \item $output$:$softmax(h)_{i}$
    \end{enumerate}
  \end{enumerate}
\end{enumerate}
\par{$maximize$:$L=\frac{1}{T}\sum_{t}\log p_{w_{t}}(C(w_{t-n}),\cdots,C(w_{t-1});\theta)+R(\theta,C)$}
\section{问题}
\begin{enumerate}
  \item 原文The Proposed Model: two Architectures中的两种架构构想中的$g$和$h$有点混乱。
\end{enumerate}

\end{document}
